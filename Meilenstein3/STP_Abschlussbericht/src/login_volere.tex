\section*{Anforderungen an das Loginsystem}
\subsection*{Funktionale Anforderungen}
\begin{myreq}
  \threeinline
    {\reqno 1}
    {\reqtype Funktional}
    {\reqevent Login}
  \reqdesc Neue Nutzer sollen sich registrieren können.
  \reqrat Die Erstellung eines Accounts ist notwendig, um nutzerspezifische Daten zuordnen zu können. 
  \reqorig Kunde 
  \reqfit Nutzername und Passwort wurden erfolgreich übermittelt und in einer Datenbank gespeichert.
  \twoinline
    {\reqsatis 2}
    {\reqdissat 4}
  \twoinline
  {\reqdep -}
  {\reqconf -}
  \reqmater -
  \reqhist 18.11.2022 - erstellt, 19.11.2022 - überarbeitet
\end{myreq}

\begin{myreq}
  \threeinline
    {\reqno 2}
    {\reqtype Funktional}
    {\reqevent Login}
  \reqdesc Bei falscher Eingabe der Logindaten soll der Nutzer aufgefordert werden Nutzername und Passwort erneut einzugeben.
  \reqrat Falsche Eingaben dürfen nicht dazu führen, dass der Login trotzdem durchgeführt wird oder der Nutzer vom Login ausgeschlossen wird.
  \reqorig Kunde
  \reqfit Die Eingabefelder für Nutzername und Passwort sind geleert und die Anwendung wartet auf erneute Eingabe dieser Daten.
  \twoinline
    {\reqsatis 1}
    {\reqdissat 5}
  \twoinline
  {\reqdep -}
  {\reqconf -}
  \reqmater -
  \reqhist 18.11.2022 - erstellt, 19.11.2022 - überarbeitet
\end{myreq}

\begin{myreq}
  \threeinline
    {\reqno 3}
    {\reqtype Funktional}
    {\reqevent Login}
  \reqdesc Man soll sich über einen Account anmelden können.
  \reqrat Um die Aufgaben lösen zu können muss man sich davor anmelden können. Außerdem wird dadurch sichergestellt, dass man seine Aufgaben selbst löst.
  \reqorig Kunde 
  \reqfit Erfüllt, wenn man sich anmelden kann.
  \twoinline
    {\reqsatis 3}
    {\reqdissat 4}
  \twoinline
  {\reqdep -}
  {\reqconf -}
  \reqmater -
  \reqhist 18.11.2022 - erstellt, 19.11.2022 - überarbeitet
\end{myreq}

\subsection*{Nichtfunktionale Anforderungen}

\begin{myreq}
  \twoinline
    {\reqno 4}
    {\reqtype Nichtfunktional}
    {\reqevent Login}
  \reqdesc Passwörter sollen verschlüsselt dargestellt werden.
  \reqrat Passwörter sollen für Andere nicht sichtbar sein.
  \reqorig Kunde
  \reqfit Die Zeichen des Passworts werden als "*" dargestellt.
  \twoinline
    {\reqsatis 3}
    {\reqdissat 4}
  \twoinline
  {\reqdep -}
  {\reqconf -}
  \reqmater -
  \reqhist 18.11.2022 - erstellt, 19.11.2022 - überarbeitet
\end{myreq}

\begin{myreq}
  \twoinline
    {\reqno 5}
    {\reqtype Nichtfunktional}
    {\reqevent Login}
  \reqdesc Die Benutzeroberfläche soll für eine unkomplizierte Nutzung intuitiv gestaltet sein.
  \reqrat Als Nutzer möchte ich mich auf die Erstellung und Bearbeitung der Aufgaben konzentrieren und keine Probleme beim Anmelden haben.
  \reqorig Kunde 
  \reqfit Erfüllt, wenn bei einem Test von zehn Personen, die sich jeweils fünfmal anmelden sollen, kein Problem aufgetreten ist.
  \twoinline
    {\reqsatis 4}
    {\reqdissat 3}
  \twoinline
  {\reqdep -}
  {\reqconf -}
  \reqmater -
  \reqhist 18.11.2022 - erstellt, 19.11.2022 - überarbeitet
\end{myreq}