%--------------------------------------------------------------------------------------%--------------------------------------------------------------------------------------
%
%  Global settings, dont change it! (excapt additional \usepackage commands)
%  Always use PDFLatex!
%
%--------------------------------------------------------------------------------------%--------------------------------------------------------------------------------------
\documentclass[a4paper, 12pt, oneside, BCOR1cm,toc=chapterentrywithdots]{scrbook}

\usepackage{graphicx}           % use for pdfLatex
\usepackage{makeidx} % f\"{u}r Benutzung des Befehls \printindex
\usepackage[colorlinks=false]{hyperref}
\usepackage{tocbibind}
\usepackage{blindtext}
\usepackage{subfigure} 
\usepackage{acronym}
\usepackage{minted}
\usepackage{multicol}
\usepackage[dvipsnames]{xcolor}
\usepackage{enumitem}
\usepackage{tcolorbox}
\usepackage{listings}

\newenvironment{myreq}[1]{%
\setlist[description]{font=\normalfont\color{darkgray}}%
\begin{tcolorbox}[colframe=black,colback=white, sharp corners, boxrule=1pt]%
\bfseries\color{blue}%
\begin{description}#1}%
{\end{description}\end{tcolorbox}}

\newcommand{\threeinline}[3]{\begin{multicols}{3}#1 #2 #3\end{multicols}}
\newcommand{\twoinline}[2]{\begin{multicols}{2}#1 #2\end{multicols}}

\newcommand{\reqno}{\item[Requirement-ID:]}
\newcommand{\reqtype}{\item[Requirement Type:]}
\newcommand{\reqevent}{\item[Event:]}
\newcommand{\reqdesc}{\item[Description:]}
\newcommand{\reqrat}{\item[Rationale:]}
\newcommand{\reqorig}{\item[Originator:]}
\newcommand{\reqfit}{\item[Fit Criterion:]}
\newcommand{\reqsatis}{\item[Customer Satisfaction:]}
\newcommand{\reqdissat}{\item[Customer Dissatisfaction:]}
\newcommand{\reqdep}{\item[Dependencies:]}
\newcommand{\reqconf}{\item[Conflicts:]}
\newcommand{\reqmater}{\item[Materials:]}
\newcommand{\reqhist}{\item[History:]}




\hypersetup{%
bookmarksnumbered=true, hyperindex=true,
%
%Im Acrobat Reader Subtitel 1. Ebene anzeigen
bookmarksopen=true, bookmarksopenlevel=1,
%
pdfborder=0 0 0 % Keine Box um die Links!
}

% --------------------------------------------------------------
% Force Tables and List to be added in Table of Content
% --------------------------------------------------------------

\renewcommand*{\tableofcontents}{%
  	\begingroup
  	\tocsection
  	\tocfile{\contentsname}{toc}
  	\endgroup
}
\renewcommand*{\listoffigures}{%
  	\begingroup
  	\tocsection
  	\tocfile{\listfigurename}{lof}
  	\endgroup
}
\renewcommand{\listoftables}{
	\begingroup
	\tocsection
	\tocfile{\listtablename}{lot}
	\endgroup
}
\begin{document}

%--------------------------------------------------------------------------------------%--------------------------------------------------------------------------------------
%
%  Here starts the userspace !
%
%--------------------------------------------------------------------------------------%--------------------------------------------------------------------------------------

%--------titlepage
\begin{titlepage}

{
    \begin{center}
        \raisebox{-1ex}{\includegraphics[scale=1.5]{TU_Chemnitz_positiv_gruen.pdf}}\\
    \end{center}
    \vspace{0.5cm}
}

\begin{center}

\LARGE{\textbf{Titel des Abschlussberichts}}\\
\vspace{1cm}


\Large{\textbf{Softwaretechnikpraktikum - Abschlussbericht}}\\ 
\vspace{1cm}

Fakultät für Informatik\\
Professur Softwaretechnik
\end{center}
\vspace{3cm}
Eingereicht von: Gruppe X (Mitglied 1, Mitglied 2...)\\
Einreichungsdatum: 12.12.1212\\
\vspace{0.3cm}\\
Betreuerin: Prof. Dr. Janet Siegmund \\
Betreuer: Dominik Gorgosch

\end{titlepage}

%---------------------------------------------------------
% Table of Contents, List of figures, List of Tables
%---------------------------------------------------------

\renewcommand{\contentsname}{Inhaltsverzeichnis}
\renewcommand{\listfigurename}{Abbildungsverzeichnis}
\renewcommand{\listtablename}{Tabellenverzeichnis}
\renewcommand{\figurename}{Abbildung}
\tableofcontents
\listoffigures
\listoftables

%---------------------------------------------------------
% Abkürzungsverzeichnis
%---------------------------------------------------------
\twocolumn
\addchap{Abkürzungsverzeichnis}
\begin{acronym}[Bash]
 \acro{TUC}{TU Chemnitz}

\end{acronym}

\onecolumn
%---------------------------------------------------------
% Here starts the real work
%---------------------------------------------------------

\chapter{Einführung}
% example of acronym
Hier sehen Sie, wie Sie eine Abkürzung nutzen können. Dieses Modul findet an der \ac{TUC} statt.
  % Load Data from File intro.tex

\chapter{Tabellen}
\input{src/example_tables} % Load Data from File example_tables

\chapter{Abbildungen}
% Example of, how to use figures

Hier sehen Sie die Einbindung von Abbildungen/Grafiken/Bildern.

\begin{figure}[h]
\centering
\includegraphics[width=0.6\textwidth]{TU_Chemnitz_positiv_gruen.pdf}
\caption{Abbildung 1}
\label{fig:pic0}
\end{figure}

Wenn Sie mehrere Bilder nebeneinander einfügen wollen, können Sie eine subfigure oder minipage nutzen.

\begin{figure}[h]
    \subfigure[Dies ist die erste Grafik]{\includegraphics[width=0.49\textwidth]{TU_Chemnitz_positiv_gruen.pdf} \label{fig:pic1}}
    \subfigure[Dies ist die zweite Grafik]{\includegraphics[width=0.49\textwidth]{TU_Chemnitz_positiv_gruen.pdf}\label{fig:pic2}}
\caption{Dies ist die Beschriftung der gesamten Grafik}
\end{figure}

Für die Referenzierung eines Bildes, nutzen Sie folgenden Befehl~\ref{fig:pic1}.
 % Load Data from File example_figures

\chapter{Codebeispiel}
%how to implement code
%https://de.overleaf.com/learn/latex/Code_Highlighting_with_minted

Im folgenden Kapitel sehen Sie, wie Sie Code in Latex einbinden können. Eine weitere Option wäre das Package listings. Zuerst sehen Sie ein Python-Beispiel.

\begin{minted}{python}
def addNum(list):
    sum = 0
    
    for val in list:
        sum = sum+val
        
    return sum
\end{minted}

Hier sehen Sie ein Java-Beispiel:

\begin{minted}{java}
public class Calculator {
    public static int addNum(int[] list) {
          int sum = 0;

        for(int i = 0; i < list.length; i++) {
            sum = sum + list[i];
        }

        return sum;
    }

    public static void main(String[] args) {
        int[] numbers = {1,2,3,4,5,6,7,8,9};
        int sum = addNum(numbers);
        System.out.printf("Sum: %d", sum);
    }
}
\end{minted}


Wenn Sie ihren Code referenzieren möchten, dann müssen Sie Listings nutzen. Dies geht auch zusammen mit minted. 

\begin{listing}
\begin{minted}[mathescape,
               linenos,
               numbersep=5pt,
               gobble=2,
               frame=lines,
               framesep=2mm]{java}
    public class Example {
        public static void main(String[] args) {
            int[] array = {1, 5, 3, 2, 4};
            
            for (int i = 0; i < array.length; i++) {
                System.out.println(array[i]);
            }
        }
    }
\end{minted}
\caption{Array in Java}
\label{java_example}
\end{listing}
\newpage

In line~2 of listing \ref{java_example}, we can see that...

%import code from file:
%\inputminted{python}{src/example.py} % Load Data from File example_code

\chapter{Abschnitte und Zitieren} % Load Data from File example_sections
\section{Abschnitt 1}
Nach McCracken et al. bestehen folgende Probleme ... bei Programmieranfängern~\cite{McCracken2001}.

\section{Abschnitt 2}
\subsection{Unterabschnitt 2.1}
\subsection{Unterabschnitt 2.2}



\chapter{Volere} % Load Data from File volere_template
\input{src/volere_template}


%---------------------------------------------------------
% bibliography based on Springer Design
%---------------------------------------------------------

\bibliographystyle{splncs03}
\renewcommand{\bibname}{Literaturverzeichnis}
\bibliography{bibliography}

\printindex

\end{document}
