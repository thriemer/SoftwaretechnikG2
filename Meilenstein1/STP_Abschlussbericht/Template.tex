%--------------------------------------------------------------------------------------%--------------------------------------------------------------------------------------
%
%  Global settings, dont change it! (excapt additional \usepackage commands)
%  Always use PDFLatex!
%
%--------------------------------------------------------------------------------------%--------------------------------------------------------------------------------------
\documentclass[a4paper, 12pt, oneside, BCOR1cm,toc=chapterentrywithdots]{scrbook}

\usepackage{graphicx}           % use for pdfLatex
\usepackage{makeidx} % f\"{u}r Benutzung des Befehls \printindex
\usepackage[colorlinks=false]{hyperref}
\usepackage{tocbibind}
\usepackage{blindtext}
\usepackage{subfigure} 
\usepackage{acronym}
\usepackage{minted}
\usepackage{multicol}
\usepackage[dvipsnames]{xcolor}
\usepackage{enumitem}
\usepackage{tcolorbox}
\usepackage{listings}

\newenvironment{myreq}[1]{%
\setlist[description]{font=\normalfont\color{darkgray}}%
\begin{tcolorbox}[colframe=black,colback=white, sharp corners, boxrule=1pt]%
\bfseries\color{blue}%
\begin{description}#1}%
{\end{description}\end{tcolorbox}}

\newcommand{\threeinline}[3]{\begin{multicols}{3}#1 #2 #3\end{multicols}}
\newcommand{\twoinline}[2]{\begin{multicols}{2}#1 #2\end{multicols}}

\newcommand{\reqno}{\item[Requirement-ID:]}
\newcommand{\reqtype}{\item[Requirement Type:]}
\newcommand{\reqevent}{\item[Event:]}
\newcommand{\reqdesc}{\item[Description:]}
\newcommand{\reqrat}{\item[Rationale:]}
\newcommand{\reqorig}{\item[Originator:]}
\newcommand{\reqfit}{\item[Fit Criterion:]}
\newcommand{\reqsatis}{\item[Customer Satisfaction:]}
\newcommand{\reqdissat}{\item[Customer Dissatisfaction:]}
\newcommand{\reqdep}{\item[Dependencies:]}
\newcommand{\reqconf}{\item[Conflicts:]}
\newcommand{\reqmater}{\item[Materials:]}
\newcommand{\reqhist}{\item[History:]}




\hypersetup{%
bookmarksnumbered=true, hyperindex=true,
%
%Im Acrobat Reader Subtitel 1. Ebene anzeigen
bookmarksopen=true, bookmarksopenlevel=1,
%
pdfborder=0 0 0 % Keine Box um die Links!
}

% --------------------------------------------------------------
% Force Tables and List to be added in Table of Content
% --------------------------------------------------------------

\renewcommand*{\tableofcontents}{%
  	\begingroup
  	\tocsection
  	\tocfile{\contentsname}{toc}
  	\endgroup
}
\renewcommand*{\listoffigures}{%
  	\begingroup
  	\tocsection
  	\tocfile{\listfigurename}{lof}
  	\endgroup
}
\renewcommand{\listoftables}{
	\begingroup
	\tocsection
	\tocfile{\listtablename}{lot}
	\endgroup
}
\begin{document}

%--------------------------------------------------------------------------------------%--------------------------------------------------------------------------------------
%
%  Here starts the userspace !
%
%--------------------------------------------------------------------------------------%--------------------------------------------------------------------------------------

%--------titlepage
\begin{titlepage}

{
    \begin{center}
        \raisebox{-1ex}{\includegraphics[scale=1.5]{TU_Chemnitz_positiv_gruen.pdf}}\\
    \end{center}
    \vspace{0.5cm}
}

\begin{center}

\LARGE{\textbf{Abschlussbericht Meilenstein 1}}\\
\vspace{1cm}


\Large{\textbf{Softwaretechnikpraktikum - Abschlussbericht}}\\ 
\vspace{1cm}

Fakultät für Informatik\\
Professur Softwaretechnik
\end{center}
\vspace{3cm}
Eingereicht von: Gruppe 2 (Lisa Neuhaus, Dai Yun, Li Haowei, Thriemer Linus, Rollhagen Leon, Liang Fudong, Barakat Hamza Adnan Daoud)\\
Einreichungsdatum: 19.11.2022\\
\vspace{0.3cm}\\
Betreuerin: Prof. Dr. Janet Siegmund \\
Betreuer: Dominik Gorgosch

\end{titlepage}
%---------------------------------------------------------
% Here starts the real work
%---------------------------------------------------------

\chapter{Bericht}
\section*{Was ist passiert}

Begonnen hat unsere Arbeit mit dem Kundengespräch, bei dem wir die Möglichkeit hatten offene Fragen zu klären. Danach haben wir uns als Gruppe organisiert und einen Termin für das erste Treffen gesucht. Bei diesem haben wir die Aufgaben untereinander verteilt. 
Da nicht alle Gruppenmitglieder zum ersten Meeting erschienen sind, hat der Projektmanager diese Personen speziell kontaktiert um Verfügbarkeiten und Aufgaben zu klären. Dabei hat sich herausgestellt, dass ein Kommilitone das Praktikum erst später abschließen möchte und hat deswegen die Gruppe verlassen.
Dann haben wir uns mit den Aufgaben beschäftigt und konnten dann beim zweiten Treffen Fragen klären, die aufgekommen sind. Zum einen, haben wir geklärt wie die einzelnen Felder ausgefüllt werden müssen und zum anderen haben wir die Bedeutung von Satisfaction und Dissatisfaction spezifiziert. 
Im letzten Meeting haben wir unsere Ausarbeitungen präsentiert und die fehlenden Karten gemeinsam gefüllt. Im Anschluss haben sich die zwei Präsentatoren nochmal getroffen um den Karten den letzten Feinschliff zu verleihen und diesen Bericht zu schreiben.

\section*{Herausforderungen und Probleme}
Das Hauptproblem war einen Termin zu finden, an dem alle Gruppenmitglieder Zeit haben. Bis jetzt haben wir noch keinen passenden Termin gefunden, an dem alle Zeit haben.
Im Allgemeinen ist die Kommunikation sehr träge, weswegen solche einfachen Fragen erst Tage später beantwortet werden. Wir werden aber trotzdem weiterhin versuchen einen gemeinsamen Termin zu finden. Zur Not auch Online oder alternativ auch nur mit einer Teilgruppe.
Es schien auch so, als ob Freitags alle Zeit hätten, die Gruppe war jedoch trotzdem nicht vollzählig.
Ein weitere Herausforderung ist, dass wir ein sehr internationales Team sind, und deswegen die Sprachbarriere erst überwunden werden muss, bevor man über technische Details reden kann. 
Des weiteren sind einige der verwendeten Tools neu und die Teilnehmer müssen sich erst einarbeiten.  Dabei sind die Meetings nützlich, weil erfahrenere Mitglieder die anderen unterstützen können.

\section*{Was lief gut}
Trotz der Herausforderungen wurde alle Aufgaben erfüllt. Hilfreich waren die Treffen, bei denen man gesehen hat, dass alle anwesenden Motiviert waren die Hürden zu meistern. Man konnte sich auf alle aktiven Mitglieder verlassen, da die Lösungen zum Stichtag präsentiert wurden.

\section*{Erkenntnisse}
Rückblickend war die Anzahl an Treffen optimal, es wird sich jedoch herausstellen, ob zweimal die Woche für den nächsten Meilenstein noch ausreicht.

Wir bevorzugen Präsenztreffen, weil man leichter Kommunizieren kann. Da die Latenz nicht vorhanden ist, muss man nicht immer warten bevor man antwortet und es passiert nicht so oft, dass man gleichzeitig anfängt zu sprechen. Zusätzlich merkt man, ob mein gegenüber mich verstanden hat.

Außerdem haben wir festgestellt, dass wir sehr unterschiedliche Technik benutzen, was später beim aufsetzen der Entwicklungsumgebung berücksichtigt werden muss.

\section*{Wer hat was bearbeitet:}
\begin{itemize}
\item Leon: 3, 5, 10
\item Lisa: 1, 2, 3, 4,
\item Fudong & Hamza: 12, 13
\item Linus: 9, 11, 14, 15, 16
\item Aufgabenssystem haben wir als Gruppe gelöst ( ohne Li & Yun)
\item Li & Yun: haben nichts bearbeitet und waren nicht bei den Meetings anwesend
\end{itemize}

\chapter{Volere}
\section*{Anforderungen an das Loginsystem}
\subsection*{Funktionale Anforderungen}
\begin{myreq}
  \threeinline
    {\reqno 1}
    {\reqtype Funktional}
    {\reqevent Login}
  \reqdesc Neue Nutzer sollen sich registrieren können.
  \reqrat Die Erstellung eines Accounts ist notwendig, um nutzerspezifische Daten zuordnen zu können. 
  \reqorig Kunde 
  \reqfit Nutzername und Passwort wurden erfolgreich übermittelt und in einer Datenbank gespeichert.
  \twoinline
    {\reqsatis 2}
    {\reqdissat 4}
  \twoinline
  {\reqdep -}
  {\reqconf -}
  \reqmater -
  \reqhist 18.11.2022 - erstellt, 19.11.2022 - überarbeitet
\end{myreq}

\begin{myreq}
  \threeinline
    {\reqno 2}
    {\reqtype Funktional}
    {\reqevent Login}
  \reqdesc Bei falscher Eingabe der Logindaten soll der Nutzer aufgefordert werden Nutzername und Passwort erneut einzugeben.
  \reqrat Falsche Eingaben dürfen nicht dazu führen, dass der Login trotzdem durchgeführt wird oder der Nutzer vom Login ausgeschlossen wird.
  \reqorig Kunde
  \reqfit Die Eingabefelder für Nutzername und Passwort sind geleert und die Anwendung wartet auf erneute Eingabe dieser Daten.
  \twoinline
    {\reqsatis 1}
    {\reqdissat 5}
  \twoinline
  {\reqdep -}
  {\reqconf -}
  \reqmater -
  \reqhist 18.11.2022 - erstellt, 19.11.2022 - überarbeitet
\end{myreq}

\begin{myreq}
  \threeinline
    {\reqno 3}
    {\reqtype Funktional}
    {\reqevent Login}
  \reqdesc Man soll sich über einen Account anmelden können.
  \reqrat Um die Aufgaben lösen zu können muss man sich davor anmelden können. Außerdem wird dadurch sichergestellt, dass man seine Aufgaben selbst löst.
  \reqorig Kunde 
  \reqfit Erfüllt, wenn man sich anmelden kann.
  \twoinline
    {\reqsatis 3}
    {\reqdissat 4}
  \twoinline
  {\reqdep -}
  {\reqconf -}
  \reqmater -
  \reqhist 18.11.2022 - erstellt, 19.11.2022 - überarbeitet
\end{myreq}

\subsection*{Nichtfunktionale Anforderungen}

\begin{myreq}
  \twoinline
    {\reqno 4}
    {\reqtype Nichtfunktional}
    {\reqevent Login}
  \reqdesc Passwörter sollen verschlüsselt dargestellt werden.
  \reqrat Passwörter sollen für Andere nicht sichtbar sein.
  \reqorig Kunde
  \reqfit Die Zeichen des Passworts werden als "*" dargestellt.
  \twoinline
    {\reqsatis 3}
    {\reqdissat 4}
  \twoinline
  {\reqdep -}
  {\reqconf -}
  \reqmater -
  \reqhist 18.11.2022 - erstellt, 19.11.2022 - überarbeitet
\end{myreq}

\begin{myreq}
  \twoinline
    {\reqno 5}
    {\reqtype Nichtfunktional}
    {\reqevent Login}
  \reqdesc Die Benutzeroberfläche soll für eine unkomplizierte Nutzung intuitiv gestaltet sein.
  \reqrat Als Nutzer möchte ich mich auf die Erstellung und Bearbeitung der Aufgaben konzentrieren und keine Probleme beim Anmelden haben.
  \reqorig Kunde 
  \reqfit Erfüllt, wenn bei einem Test von zehn Personen, die sich jeweils fünfmal anmelden sollen, kein Problem aufgetreten ist.
  \twoinline
    {\reqsatis 4}
    {\reqdissat 3}
  \twoinline
  {\reqdep -}
  {\reqconf -}
  \reqmater -
  \reqhist 18.11.2022 - erstellt, 19.11.2022 - überarbeitet
\end{myreq}
\section*{Anforderungen an das Aufgabensystem}
\subsection*{Funktionale Anforderungen}
\begin{myreq}
  \threeinline
    {\reqno 6}
    {\reqtype Funktional}
    {\reqevent Aufgabe lösen}
  \reqdesc Es soll Eingabemasken für alle Aufgabentypen geben.
  \reqrat Die Benutzer sollen alle Aufgabentypen lösen können.
  \reqorig Kunde
  \reqfit Alle Aufgabenmasken wurden über Unit- und Integrationstests überprüft.
  \twoinline
    {\reqsatis 2}
    {\reqdissat 4}
  \twoinline
  {\reqdep -}
  {\reqconf -}
  \reqmater -
  \reqhist 18.11.2022 - erstellt, 19.11.2022 - überarbeitet
\end{myreq}


\begin{myreq}
  \threeinline
    {\reqno 7}
    {\reqtype Funktional}
    {\reqevent Internetabbruch abfangen}
  \reqdesc Die Eingaben der Benutzer werden aller zehn Sekunden im System gespeichert.
  \reqrat Bei Intenernetabbrüchen ist es benutzerfreundlich, wenn nicht alle Eingaben verloren gehen.
  \reqorig Kunde
  \reqfit Diese Anforderung kann man mit automatisierten Tests überprüfen.
  \twoinline
    {\reqsatis 4}
    {\reqdissat 3}
  \twoinline
  {\reqdep -}
  {\reqconf -}
  \reqmater -
  \reqhist 18.11.2022 - erstellt, 19.11.2022 - überarbeitet
\end{myreq}

\subsection*{Nichtfunktionale Anforderungen}

\begin{myreq}
  \twoinline
    {\reqno 8}
    {\reqtype Nicht-Funktional}
    {\reqevent Aufgabe lösen}
  \reqdesc Eingabemasken haben Function zur Formatierung und Syntaxhighlighting.
  \reqrat Code wird durch Syntaxhighlights und Formatierung deutlich besser lesbar.
  \reqorig Kunde
  \reqfit Diese Anforderung kann man über manuelle Tests überprüfen.
  \twoinline
    {\reqsatis 4}
    {\reqdissat 3}
  \twoinline
  {\reqdep -}
  {\reqconf -}
  \reqmater -
  \reqhist 18.11.2022 - erstellt, 19.11.2022 - überarbeitet
\end{myreq}


%Quelle: https://tex.stackexchange.com/questions/515525/how-to-make-a-custom-requirement-card-like-voleres

\begin{myreq}
  \threeinline
    {\reqno 75}
    {\reqtype Funktional}
    {\reqevent -}
  \reqdesc description text description text description text 
  \reqrat some more text some more text some more text 
  \reqorig other text other text other text 
  \reqfit longer text that needs more than one line longer text that needs more than one line
  \twoinline
    {\reqsatis 5}
    {\reqdissat 3}
  \twoinline
  {\reqdep some more text}
  {\reqconf 111}
  \reqmater other text other text other text 
  \reqhist other text other text other text 
\end{myreq}
%Quelle: https://tex.stackexchange.com/questions/515525/how-to-make-a-custom-requirement-card-like-voleres

\begin{myreq}
  
    \reqno 75
    \reqtype nicht  funktionale anforderungung
    \reqevent 7.9
 
  \reqdesc Die Auswertungsdatein werden gesamelt.
  \reqrat Die Benutzer können besser erfahren,welche fehler
Sie gemacht haben.Das datenbank zeigt , wie viel Fehler
Sie gemacht haben, und welche Aufgaben schwer oder einfach
war.
   
  \reqorig 
  \reqfit Achtzig prozent der lösungen zu den Aufgaben müssen gespeichert un angezeigt werden.
  \twoinline
    \reqsatis 5
    \reqdissat 3
  \reqdep 
  \reqconf
  \reqmater other text other text other text 
  \reqhist 16.11.2022
\end{myreq}


\begin{myreq}
  
    \reqno 75
    \reqtype funktionale anforderungung
    \reqevent 7.9
 
  \reqdesc Das system muss verschiedene Aufgabentypen auswerten.
  \reqrat Das Aufgabensystem besteht aus verschiedene Aufgabentypen,
die Jeweils einen einen speyiellen regeln zur
auswertung benotigen
  \reqorig 
  \reqfit Sieben verschiedenen Aufgabenarten mussen bewertet werden.
  \twoinline
    \reqsatis 5
    \reqdissat 3
  \reqdep 
  \reqconf
  \reqmater other text other text other text 
  \reqhist 16.11.2022
\end{myreq}

\begin{myreq}
  
    \reqno 75
    \reqtype funktionale anforderungung
    \reqevent 7.9
 
  \reqdesc das system muss Java und Python emulieren
  \reqrat alle Aufgaben Arten im Aufgabesystemteil testen die
Java und python kenntnisse fur den benutzer
  \reqorig 
  \reqfit mindestens eine Aufgabentyp,der mit java oder python geschrieben wurde, sollte auswerten werden.
  \twoinline
    \reqsatis 5
    \reqdissat 3
  \reqdep 
  \reqconf
  \reqmater other text other text other text 
  \reqhist 16.11.2022
\end{myreq}


\end{document}
