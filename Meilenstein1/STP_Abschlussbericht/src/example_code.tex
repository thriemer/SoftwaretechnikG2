%how to implement code
%https://de.overleaf.com/learn/latex/Code_Highlighting_with_minted

Im folgenden Kapitel sehen Sie, wie Sie Code in Latex einbinden können. Eine weitere Option wäre das Package listings. Zuerst sehen Sie ein Python-Beispiel.

\begin{minted}{python}
def addNum(list):
    sum = 0
    
    for val in list:
        sum = sum+val
        
    return sum
\end{minted}

Hier sehen Sie ein Java-Beispiel:

\begin{minted}{java}
public class Calculator {
    public static int addNum(int[] list) {
          int sum = 0;

        for(int i = 0; i < list.length; i++) {
            sum = sum + list[i];
        }

        return sum;
    }

    public static void main(String[] args) {
        int[] numbers = {1,2,3,4,5,6,7,8,9};
        int sum = addNum(numbers);
        System.out.printf("Sum: %d", sum);
    }
}
\end{minted}


Wenn Sie ihren Code referenzieren möchten, dann müssen Sie Listings nutzen. Dies geht auch zusammen mit minted. 

\begin{listing}
\begin{minted}[mathescape,
               linenos,
               numbersep=5pt,
               gobble=2,
               frame=lines,
               framesep=2mm]{java}
    public class Example {
        public static void main(String[] args) {
            int[] array = {1, 5, 3, 2, 4};
            
            for (int i = 0; i < array.length; i++) {
                System.out.println(array[i]);
            }
        }
    }
\end{minted}
\caption{Array in Java}
\label{java_example}
\end{listing}
\newpage

In line~2 of listing \ref{java_example}, we can see that...

%import code from file:
%\inputminted{python}{src/example.py}